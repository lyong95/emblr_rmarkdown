% Options for packages loaded elsewhere
\PassOptionsToPackage{unicode}{hyperref}
\PassOptionsToPackage{hyphens}{url}
%
\documentclass[
]{article}
\usepackage{amsmath,amssymb}
\usepackage{lmodern}
\usepackage{ifxetex,ifluatex}
\ifnum 0\ifxetex 1\fi\ifluatex 1\fi=0 % if pdftex
  \usepackage[T1]{fontenc}
  \usepackage[utf8]{inputenc}
  \usepackage{textcomp} % provide euro and other symbols
\else % if luatex or xetex
  \usepackage{unicode-math}
  \defaultfontfeatures{Scale=MatchLowercase}
  \defaultfontfeatures[\rmfamily]{Ligatures=TeX,Scale=1}
\fi
% Use upquote if available, for straight quotes in verbatim environments
\IfFileExists{upquote.sty}{\usepackage{upquote}}{}
\IfFileExists{microtype.sty}{% use microtype if available
  \usepackage[]{microtype}
  \UseMicrotypeSet[protrusion]{basicmath} % disable protrusion for tt fonts
}{}
\makeatletter
\@ifundefined{KOMAClassName}{% if non-KOMA class
  \IfFileExists{parskip.sty}{%
    \usepackage{parskip}
  }{% else
    \setlength{\parindent}{0pt}
    \setlength{\parskip}{6pt plus 2pt minus 1pt}}
}{% if KOMA class
  \KOMAoptions{parskip=half}}
\makeatother
\usepackage{xcolor}
\IfFileExists{xurl.sty}{\usepackage{xurl}}{} % add URL line breaks if available
\IfFileExists{bookmark.sty}{\usepackage{bookmark}}{\usepackage{hyperref}}
\hypersetup{
  pdftitle={rmarkdown},
  pdfauthor={Luoyan Yong},
  hidelinks,
  pdfcreator={LaTeX via pandoc}}
\urlstyle{same} % disable monospaced font for URLs
\usepackage[margin=1in]{geometry}
\usepackage{color}
\usepackage{fancyvrb}
\newcommand{\VerbBar}{|}
\newcommand{\VERB}{\Verb[commandchars=\\\{\}]}
\DefineVerbatimEnvironment{Highlighting}{Verbatim}{commandchars=\\\{\}}
% Add ',fontsize=\small' for more characters per line
\usepackage{framed}
\definecolor{shadecolor}{RGB}{248,248,248}
\newenvironment{Shaded}{\begin{snugshade}}{\end{snugshade}}
\newcommand{\AlertTok}[1]{\textcolor[rgb]{0.94,0.16,0.16}{#1}}
\newcommand{\AnnotationTok}[1]{\textcolor[rgb]{0.56,0.35,0.01}{\textbf{\textit{#1}}}}
\newcommand{\AttributeTok}[1]{\textcolor[rgb]{0.77,0.63,0.00}{#1}}
\newcommand{\BaseNTok}[1]{\textcolor[rgb]{0.00,0.00,0.81}{#1}}
\newcommand{\BuiltInTok}[1]{#1}
\newcommand{\CharTok}[1]{\textcolor[rgb]{0.31,0.60,0.02}{#1}}
\newcommand{\CommentTok}[1]{\textcolor[rgb]{0.56,0.35,0.01}{\textit{#1}}}
\newcommand{\CommentVarTok}[1]{\textcolor[rgb]{0.56,0.35,0.01}{\textbf{\textit{#1}}}}
\newcommand{\ConstantTok}[1]{\textcolor[rgb]{0.00,0.00,0.00}{#1}}
\newcommand{\ControlFlowTok}[1]{\textcolor[rgb]{0.13,0.29,0.53}{\textbf{#1}}}
\newcommand{\DataTypeTok}[1]{\textcolor[rgb]{0.13,0.29,0.53}{#1}}
\newcommand{\DecValTok}[1]{\textcolor[rgb]{0.00,0.00,0.81}{#1}}
\newcommand{\DocumentationTok}[1]{\textcolor[rgb]{0.56,0.35,0.01}{\textbf{\textit{#1}}}}
\newcommand{\ErrorTok}[1]{\textcolor[rgb]{0.64,0.00,0.00}{\textbf{#1}}}
\newcommand{\ExtensionTok}[1]{#1}
\newcommand{\FloatTok}[1]{\textcolor[rgb]{0.00,0.00,0.81}{#1}}
\newcommand{\FunctionTok}[1]{\textcolor[rgb]{0.00,0.00,0.00}{#1}}
\newcommand{\ImportTok}[1]{#1}
\newcommand{\InformationTok}[1]{\textcolor[rgb]{0.56,0.35,0.01}{\textbf{\textit{#1}}}}
\newcommand{\KeywordTok}[1]{\textcolor[rgb]{0.13,0.29,0.53}{\textbf{#1}}}
\newcommand{\NormalTok}[1]{#1}
\newcommand{\OperatorTok}[1]{\textcolor[rgb]{0.81,0.36,0.00}{\textbf{#1}}}
\newcommand{\OtherTok}[1]{\textcolor[rgb]{0.56,0.35,0.01}{#1}}
\newcommand{\PreprocessorTok}[1]{\textcolor[rgb]{0.56,0.35,0.01}{\textit{#1}}}
\newcommand{\RegionMarkerTok}[1]{#1}
\newcommand{\SpecialCharTok}[1]{\textcolor[rgb]{0.00,0.00,0.00}{#1}}
\newcommand{\SpecialStringTok}[1]{\textcolor[rgb]{0.31,0.60,0.02}{#1}}
\newcommand{\StringTok}[1]{\textcolor[rgb]{0.31,0.60,0.02}{#1}}
\newcommand{\VariableTok}[1]{\textcolor[rgb]{0.00,0.00,0.00}{#1}}
\newcommand{\VerbatimStringTok}[1]{\textcolor[rgb]{0.31,0.60,0.02}{#1}}
\newcommand{\WarningTok}[1]{\textcolor[rgb]{0.56,0.35,0.01}{\textbf{\textit{#1}}}}
\usepackage{longtable,booktabs,array}
\usepackage{calc} % for calculating minipage widths
% Correct order of tables after \paragraph or \subparagraph
\usepackage{etoolbox}
\makeatletter
\patchcmd\longtable{\par}{\if@noskipsec\mbox{}\fi\par}{}{}
\makeatother
% Allow footnotes in longtable head/foot
\IfFileExists{footnotehyper.sty}{\usepackage{footnotehyper}}{\usepackage{footnote}}
\makesavenoteenv{longtable}
\usepackage{graphicx}
\makeatletter
\def\maxwidth{\ifdim\Gin@nat@width>\linewidth\linewidth\else\Gin@nat@width\fi}
\def\maxheight{\ifdim\Gin@nat@height>\textheight\textheight\else\Gin@nat@height\fi}
\makeatother
% Scale images if necessary, so that they will not overflow the page
% margins by default, and it is still possible to overwrite the defaults
% using explicit options in \includegraphics[width, height, ...]{}
\setkeys{Gin}{width=\maxwidth,height=\maxheight,keepaspectratio}
% Set default figure placement to htbp
\makeatletter
\def\fps@figure{htbp}
\makeatother
\setlength{\emergencystretch}{3em} % prevent overfull lines
\providecommand{\tightlist}{%
  \setlength{\itemsep}{0pt}\setlength{\parskip}{0pt}}
\setcounter{secnumdepth}{-\maxdimen} % remove section numbering
\ifluatex
  \usepackage{selnolig}  % disable illegal ligatures
\fi

\title{rmarkdown}
\author{Luoyan Yong}
\date{27/04/2021}

\begin{document}
\maketitle

\hypertarget{three-ways-to-use-rmarkdown}{%
\subsection{Three ways to use
rmarkdown}\label{three-ways-to-use-rmarkdown}}

\begin{enumerate}
\def\labelenumi{\arabic{enumi}.}
\tightlist
\item
  Reporting conclusions without going into the code
\item
  collaboration with others who might be interested in both the
  conclusions you draw and code behind.
\item
  lab notebook to record what you did and the thinking process behind
\end{enumerate}

\hypertarget{basic-parts-of-an-rmarkdown-document}{%
\subsection{Basic parts of an rmarkdown
document}\label{basic-parts-of-an-rmarkdown-document}}

\begin{enumerate}
\def\labelenumi{\arabic{enumi}.}
\tightlist
\item
  Text with simple formatting
\end{enumerate}

\begin{longtable}[]{@{}ll@{}}
\toprule
enclose with & example \\
\midrule
\endhead
* or \_ & \emph{italics} \\
** or \_\_ & \textbf{bold} \\
\texttt{\textbar{}}nrow(mtcars)` & \\
\^{} & 10\textsuperscript{2} \\
\textasciitilde{} & log\textsubscript{10} \\
\bottomrule
\end{longtable}

{[}Rmd
cheatsheet{]}{[}\url{https://www.rstudio.com/wp-content/uploads/2016/03/rmarkdown-cheatsheet-2.0.pdf}{]}

\begin{enumerate}
\def\labelenumi{\arabic{enumi}.}
\setcounter{enumi}{1}
\item
  code chunk

  Add with one of the 3 ways:
\end{enumerate}

\begin{itemize}
\tightlist
\item
  Cmd/Ctrl + Alt + I
\item
  Insert button
\item
  typing in chunk delimiter
\end{itemize}

\begin{Shaded}
\begin{Highlighting}[]
\FunctionTok{print}\NormalTok{(}\StringTok{"I am the output of this code chunk."}\NormalTok{)}
\end{Highlighting}
\end{Shaded}

\begin{verbatim}
## [1] "I am the output of this code chunk."
\end{verbatim}

\begin{figure}
\centering
\includegraphics{/g/typas/Personal_Folders/Luoyan/emblR/Rmarkdown/chunk_options_table.png}
\caption{chunk options}
\end{figure}

\begin{Shaded}
\begin{Highlighting}[]
\FunctionTok{ggplot}\NormalTok{(mtcars, }\FunctionTok{aes}\NormalTok{(}\AttributeTok{x =}\NormalTok{ cyl, }\AttributeTok{y =}\NormalTok{ mpg)) }\SpecialCharTok{+} 
  \FunctionTok{geom\_point}\NormalTok{()}
\end{Highlighting}
\end{Shaded}

Be careful about caching + good for loading up computationally expensive
output - not based on dependencies

\begin{Shaded}
\begin{Highlighting}[]
\NormalTok{data }\OtherTok{=}\NormalTok{ mtcars[}\DecValTok{1}\SpecialCharTok{:}\DecValTok{10}\NormalTok{, ]}
\end{Highlighting}
\end{Shaded}

\begin{Shaded}
\begin{Highlighting}[]
\NormalTok{proc }\OtherTok{=}\NormalTok{ dplyr}\SpecialCharTok{::}\FunctionTok{filter}\NormalTok{(data, am }\SpecialCharTok{==} \DecValTok{1}\NormalTok{)}
\NormalTok{proc}
\end{Highlighting}
\end{Shaded}

\begin{verbatim}
##                mpg cyl disp  hp drat    wt  qsec vs am gear carb
## Mazda RX4     21.0   6  160 110 3.90 2.620 16.46  0  1    4    4
## Mazda RX4 Wag 21.0   6  160 110 3.90 2.875 17.02  0  1    4    4
## Datsun 710    22.8   4  108  93 3.85 2.320 18.61  1  1    4    1
\end{verbatim}

\begin{Shaded}
\begin{Highlighting}[]
\NormalTok{data }\OtherTok{=}\NormalTok{ mtcars[}\DecValTok{11}\SpecialCharTok{:}\DecValTok{20}\NormalTok{, ]}
\end{Highlighting}
\end{Shaded}

\begin{Shaded}
\begin{Highlighting}[]
\NormalTok{proc }\OtherTok{=}\NormalTok{ dplyr}\SpecialCharTok{::}\FunctionTok{filter}\NormalTok{(data, am }\SpecialCharTok{==} \DecValTok{1}\NormalTok{)}
\NormalTok{proc}
\end{Highlighting}
\end{Shaded}

\begin{verbatim}
##                 mpg cyl disp hp drat    wt  qsec vs am gear carb
## Fiat 128       32.4   4 78.7 66 4.08 2.200 19.47  1  1    4    1
## Honda Civic    30.4   4 75.7 52 4.93 1.615 18.52  1  1    4    2
## Toyota Corolla 33.9   4 71.1 65 4.22 1.835 19.90  1  1    4    1
\end{verbatim}

Global options defines default for every chunk

\begin{Shaded}
\begin{Highlighting}[]
\NormalTok{knitr}\SpecialCharTok{::}\NormalTok{opts\_chunk}\SpecialCharTok{$}\FunctionTok{set}\NormalTok{(}\AttributeTok{echo =} \ConstantTok{FALSE}\NormalTok{)}
\end{Highlighting}
\end{Shaded}

\begin{verbatim}
## Warning in mean.default(data): argument is not numeric or logical: returning NA
\end{verbatim}

Inline code Example: There are a total of 10 diamonds in the dataset
with an average weight of

\end{document}
